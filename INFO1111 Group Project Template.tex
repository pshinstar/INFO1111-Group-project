\documentclass[a4paper, 11pt]{report}
\usepackage{blindtext}
\usepackage[T1]{fontenc}
\usepackage[utf8]{inputenc}
\usepackage{titlesec}
\usepackage{fancyhdr}
\usepackage{geometry}

\usepackage[english]{babel}
\usepackage{apacite}

\geometry{ margin=30mm }
\counterwithin{subsection}{section}
\renewcommand\thesection{\arabic{section}.}
\renewcommand\thesubsection{\thesection\arabic{subsection}.}
\usepackage{tocloft}
\renewcommand{\cftchapleader}{\cftdotfill{\cftdotsep}}
\renewcommand{\cftsecleader}{\cftdotfill{\cftdotsep}}
\setlength{\cftsecindent}{2.2em}
\setlength{\cftsubsecindent}{4.2em}
\setlength{\cftsecnumwidth}{2em}
\setlength{\cftsubsecnumwidth}{2.5em}


\begin{document}
\titleformat{\section}
{\normalfont\fontsize{15}{0}\bfseries}{\thesection}{1em}{}
\titlespacing{\section}{0cm}{0.5cm}{0.15cm}
\titleformat{\subsection}
{\normalfont\fontsize{13}{0}\bfseries}{\thesubsection}{0.5em}{}
\titlespacing{\section}{0cm}{0.5cm}{0.15cm}

%=======================================================================================

\begin{titlepage}
\center 
\textbf{\huge INFO1111: Computing 1A Professionalism}\\[0.75cm]
\textbf{\huge 2022 Semester 1}\\[2cm]
\textbf{\huge Practice: Team Project Report}\\[3cm]

\textbf{\huge Submission number: 2}\\[0.75cm]
\textbf{\huge Team Members:}\\[0.75cm]
\textbf{\large
    \begin{tabular}{|p{0.5\textwidth}|p{0.3\textwidth}|p{0.2\textwidth}|}
        \hline
        Name & Student ID & Levels being attempted in this submission\\
        \hline
        Ali Sheng & 520470671 & 3 \\
        Dong Yoon Shin (Patrick) & 520271045 & 3 \\
        Lily Weng & 520474864 & 3 \\
        Roshni Vadassery & 520462540 & 3 \\
        \hline
    \end{tabular}
}\\[0.75cm]
\end{titlepage}

%=======================================================================================

\tableofcontents

%=======================================================================================

\newpage
\section*{General Instructions}

You should use this \LaTeX\ template to generate your team project report. Keep in mind the following key points:
\begin{itemize}
    \item When we assess your report, you are not given a mark. Instead we will indicate (separately, for each team member) whether each level is ''achieved''.
    \item In order to pass the unit, you must achieve at least level 1. 
    \item In order to achieve level 2, you must first have achieved level 1, and so on for each level up to level 4. This means that we will not assess a higher level until a lower level has been achieved (though we will review one level higher and give you feedback to help you in refining your work).
    \item Some parts of the report are completed as a team and other parts require each student to complete a different section. This means that for each submission, some members of the team may have completed their work for a given section, but other members may not. It also is therefore possible that some members of the team may achieve a specified level and other members of the team may not yet have achieved that level.
    \item Even if some members are completing their material for a given level, and others are not, your team members will still need to work together to edit and compile the report.  The only exception to this is where a member of the team has already achieved the level they are targeting in a previous submission and has decided to not attempt higher levels, and so is not contributing any further (this should be obvious because no level is indicated for that student on the cover page).
    \item When completing each section you should remove the explanation text and replace it with your material.
\end{itemize}

For each submission you will add new details to this report, and/or update previous sections (where previous work was not good enough to have achieved the relevant level). In particular:

\begin{itemize}
    \item \textbf{General:} For each submission, each student can attempt up to 2 levels. You must also successfully achieve each lower level before you can be assessed at a higher level. For example, in the first submission you might attempt only level 1, but not be successful in achieving that level. You then reattempt level 1 and add in level 2 in the second submission and are successful in achieving level 1 but not level 2. For the third and final submission you could then attempt level 2, or levels 2 and 3 - or even just choose to not submit anything further and remain at level 1).
    \item \textbf{Submission 1:} You should complete at least the material for level 1 (since achieving level 1 is required to pass the unit). Each member of the team can also optionally choose to complete the material for level 2.\\
    \textit{Note 1: If you do not complete the level 2 information then you obviously cannot achieve level 2 at this stage. This does not stop you from attempting level 2 in Deliverable 2 or 3, but it will make it more difficult to achieve the higher levels later in the semester.}
    \textit{Note 2: To be able to achieve Level 1 in submission one your team has to achieve level 1 in the group component (Section 1.1) and you have to achieve Level 1 in the individual component (i.e. your assigned section 1.2, 1.3, 1.4 or 1.5)}
    \item \textbf{Submission 2:} Each member of your team will complete additional sections, but because you are submitting a single document, you need to work together to compile your results together and generate the final submission.\\
    If you did not achieve level 1 in your first submission, then you should revise the material for level 1 based on the feedback, and optionally you can also complete level 2.\\
    If you achieved level 1 in your first submission, then each team member can optionally complete the material for levels 2 and 3.
    \textit{Note: If you do not achieve level 1 with this submission then the highest level you will be able to achieve in the final submission will be level 2. If you achieve level 1, but not level 2, with this submission then the highest level you will be able to achieve with the final submission is level 3.}
    \item \textbf{Submission 3:} Again, you can correct sections where you did not achieve the specified level in the previous submission, and you complete additional sections.\\
    If you still have not achieved level 1, then you should revise the material for level 1 based on the feedback, and again optionally you can also complete level 2.\\
    For those at level 1, you can choose to complete the material for levels 2 and 3.\\
    For those at level 2, you can choose to complete the material for levels 3 and 4.\\
    For those at level 3, you can choose to complete the material for level 4.
\end{itemize}

Whilst the team project is just that -- a team project -- it has been designed to also allow different members of the team to achieve different outcomes. We do expect you to work together as a team. If you do come across problems working together then the first step should be to discuss this with your tutor. Note: If you are having problems you should approach your tutor as soon as you can to make them aware of the difficulties you are having with your team.

Finally, you should also ensure that any resources you use are suitably referenced, and references are included into the reference list at the end of this document. You should use APA 6th reference style \cite{apa6}.

%=======================================================================================

\newpage
\section{Level 1: Basic Skills}

Level 1 focuses on basic technical skills (related to \LaTeX\ and Git) and the types of skills used in different computing jobs.

\subsection{Developing industry skills}

This section is completed as a team.\\
Throughout your Computing degree we will help you learn a range of new skills. Once you graduate however you will need to continue to learn new languages, new tools, new applications, etc. For this section you need to identify 5 approaches you can take to this continual learning. You should then put these in order from most effective to least effective, and then explain the circumstances in which each approach might be appropriate. (Target = $\sim$100 words per skill = $\sim$500 words total).

\textbf{Self learning}
Working in computer science requires the attitude of continual learning to be aware and up to date with the rapidly evolving innovations relating to technology. This allows employees to continue maintaining their valuable status for the company and prevent skills from becoming outdated. It is common for all workers in computer science to execute self learning to some level as it shows the employer that the candidate has full capability to perform personal projects. It also shows the employer the candidate has significant desire to learn, creating an impressive impression when choosing successful candidates. Therefore the positive action of continual self learning in the computer science industry is seen as an essential and required skill to have, as it maintains their skills that are applicable with the latest technologies and also generates a positive impression amongst employers

\textbf{Internal training and external training}
Internal training opens the opportunity to learn from the more experienced employees to continue perfecting and advancing an individual's own skill. It is not only advantageous for the employee receiving the training, but it can also reinforce the trainer's knowledge to motivate them to continue learning as well. Internal training is an easy and affordable method for continual learning while working in computer science. External training also has many benefits in offering employees training courses to continue learning. 
External training opens up the company’s employees and teams to industry experts, offering a new perspective on different matters related to computer science. This can contribute to the company standing out from others as employees develop the skill of versatility from this training. 

\textbf{Efficient Workspace}

With the field of computing perpetually accelerating with new technology, a high level of self-discipline and autonomy is required to maintain a solid level of professionalism and to keep learnt skills in practice. As working at home has become a common phenomenon after 2020s, constructing an efficient workspace is key to improving productivity. Usually, keeping distractions out of reach and building a strong and focused mindset is the classic go-to, but that requires much self-control and would only come with time. Some other options would be going to cafes, libraries and even shared workspaces, as being in an environment where others are working encourages concentration.

\textbf{Personal Projects}

Besides being a library of codes displaying the skillset of a computer scientist, personal projects assist in the important task of keeping the myriad of languages used in practice. While everyone has their own opinion on the best language e.g. Python, C/C++, JavaScript for whatever task at hand, their variety of contradicting syntax requires practice to prevent re-learning of the language, wasting time and stalling advancement. Furthermore, the best way to show yourself and potential employers the level of your abilities in a particular coding language and particular field is to create a personal project.

\textbf{Development Goal}

  Computer-related disciplines require constant development objectives and a sense of progress for vision as the world's IT-related technologies advance. This necessitates setting personal goals apart from those set by the industry. Humans can only attain their original efficiency if they have a purpose, just as a goal attracts all energy like a magnet. Furthermore, having a defined aim allows the developer to have a clear direction and focus on the tasks at hand, increasing the efficiency of work and the quality of the outcomes. Furthermore, it enables to maintain a good attitude and approach challenging challenges in a variety of ways. A sense of achievement makes to set higher goals, which finally gain the greater advanced computer-related skills will be, which leads to success in industry.



\subsection{Skills: Dong Yoon Shin (Patrick) : Computer Science}

\textbf{Understanding Human Error}
  While human process machine learning, programming, and creating algorithms, resulting in severe system dangers. Human error is a significant factor to consider in complicated safety-critical systems since it contributes the most to overall system risk \cite{hum}. Mistakes can never be eliminated since, in comparison to computers, the human aspect makes forecasting, avoiding, and unpredictability more difficult. For example, collected data, coding. Understanding how human error occurs helps to the prevention of large system risks and the reduction of human-induced errors, such as those made by co-workers in industry. As a result, more qualified outcomes will be obtained with fewer significant problems.

\textbf{Creativity}
  Professional computer scientists demand creativity to break free from old patterns or habits of thought and open up new possibilities. Unlike mathematicians, computer scientists must be able to benefit from an infinite number of possibilities and solution paths, which necessitates inventiveness. Most modern companies require someone who can think of new ways to expand existing solutions, such as algorithms. Innovation leads to the development of all computer systems in an industry as a result of creative ideas. Creativity that can be paired with technology that can handle them freely is necessary in the reconfiguration and exploitation of current computer systems., especially to construct and deal with new systems, software, requires logical thinking and inventiveness.

\textbf{Mathematics Skills}
  Computer scientists must have a good understanding of mathematical theory, including calculus, statistics, linear algebra, some physics concepts, and any other advanced mathematical methods, as mathematics is a general necessity for any computer factor. This is because, while mathemtics is an important intellectual tool in programming, computing is becoming increasingly important in solving mathematical issues. Furthermore, Mathematical Concepts are Required in Many Computer Science Disciplines, such as the use of algorithms and the provision of Analytical Skills. As a result, mathemtics abilities are required to comprehend computer systems used in the modern IT business, such as those used by Google and Microsoft.

\textbf{Programming Skills}
  Python, R, Java, Javascirpt, C, and more programming languages are available to computer scientists. Computer scientists research the theory of computation and the design of software systems, hence they are in charge of investigating these processes from a computer science standpoint. They must, however, be able to solve problems with computers and apply solutions while coding. The primary function of a computer scientist is design, not construction. To build an algorithm, computer scientist need to be familiar with a variety of programming languages. In addition, by learning computer languages, you will be able to assist teams with their responsibilities.

\textbf{Analytical Skills}
  To think critically, examine data, make difficult judgments, and solve complicated issues, computer scientists need great analytical skills. \cite{ident} A computer scientist's main task is to identify a problem and design a solution to address it. This necessitates strong analytical abilities, which will allow computer sicentists to grasp the situation at hand and evaluate numerous options before selecting the one that best fit the needs of all stakeholders. Computer scientists utilize two analytical skills during the task data analysis and experimenting with technical applications. These abilities result in accurate data and outcomes, as well as the greatest results for clients and businesses.

\textbf{Problem Solving Skills}
  One of the most important skills for computer scientists is problem solving. Majors in computer science must be able to tackle complicated issues in a systematic and logical manner, which necessitates the use of mathematics, creativity, and other skills. The reason for this is that a computer scientist must select which project he or she is working on and how to take a concept and turn it into a reality. For this, computer scientists must be able to comprehend problems that can be solved by humans and computers, as well as write specialised syntax to solve programs \cite{ident}. This can assist the team in determining what they need to do, as well as what tools and resources they will require. Overall, the team's efficiency has improved.

\textbf{Communication}
  Communication is a skill that is necessary in all industries. Good communication skills are required for the progress that the stakeholder desires. This is because the talent allows people to utilise listening and questioning to work through concerns, difficulties, and challenges relating to IT initiatives. Constant communication is essential because it allows for assistance in explaining projects, objectives, and timelines to clients, coworkers, and managers, which leads to efficient work and desired success by communicating about a design of algorithm and realistic problems for the project, whereas miscommunication can lead to confusion and errors, which reduce efficiency \cite{comu}. As a result, communication assists in the achievement of the project's objectives and gives a more consistent response to the company's or customers' demands.

\textbf{Technical Writing}
  Briefs, proposals, reports, and other important technical documents require computer scientists to be able to communicate effectively in writing\cite{tcw}. By writing documentations of overall project reports, the team may choose the next action to take by simplifying and visualising difficult information. This makes it easier to comprehend and solve difficulties. With technical writing, programmers will be able to provide a blueprint for the algorithms and steps of software. Furthermore, depending on the drafting of the report, a time schedule and a To-Do list may be created, resulting in a better understanding. As a result, better time management, faster production, and fewer misunderstandings will occur, resulting in industrial failure.


\subsection{Skills: Li Sheng : Data Science}


\textbf{Communication}
While strong communication skills are crucial to essentially all lines of work, I feel it is especially important to data scientists. The role requires communicating information to team members/partners that may not have a data/computing background, as well as needing to understand the ‘problem’ that needs to be solved. As data scientists would usually work in teams, good communication would allow them to mesh into various team dynamics easily. Furthermore, as pure intuition would only go so far, good interpersonal skills would help probe stakeholders for more data, information, and key insights.

\textbf{Business Acumen}
Adding onto communication skills, business knowledge adds to the effectiveness of communication, making the process of problem solving much more efficient. As data scientists usually work to solve issues for businesses, business insight assists in understanding the key objectives and goals of the business, and thus the ability to create cost effective, easy to implement solutions which would ensure broad adoption \cite{ali2}. Furthermore, it is helpful in promoting business growth and exploring new business opportunities from a more data focused angle.

\textbf{Critical/Structured Thinking}
The core of data science requires deep curiosity and critical thinking, to solve problems and to find solutions, as data on its own does not mean much without processing and interpretation from data scientists. Alongside critical thinking, storytelling skills are also important, for example, the score of a live basketball game, on one side, it is displayed in the form of a table, the other side, as a bar chart, whereas the bar chart obviously being more intuitive, as it is human nature to prefer a more visual medium e.g. infographics, and it being the job of the data scientist display intricate data in a medium digestible to the average person.

\textbf{Programming}
Being a data scientist means communicating with machines, with programming languages such as Python, R and C/C++ as their tools. While they do not need to become the best at programming like software engineers do, they will need to be comfortable with it. The main role of programming languages for data scientists is to organise, analyse and visualise data, so having a library of different languages helps tackle this task from multiple angles. This also assists in filling basic team roles e.g. programming, frontend/backend engineering, which all computing roles would entail.

\textbf{Machine Learning/Deep Learning}
For data scientists working with mountains of organised and unorganised data, machines/deep learning builds predictive models and algorithms which effectively streamlines the organisation and prediction of data \cite{ali1}. This relatively emerging school of technology is not necessary for a data scientist but is a skill which would improve employability by automating the entire process by cleaning data and removing redundancies. Machine learning is also extremely malleable, being used in Fraud/Risk Detection Management, Healthcare, Airline Route Planning, Spam Filtering, Facial/Voice Recognition System, Interactive Voice Response, Comprehensive Language and Document Recognition and Translation.

\textbf{Big Data}
The growing ubiquity of the internet, social media and IoT has led to a sudden boom in the rate of data being generated, being up to 2.5 quintillion per day, new and innovative methods are required for data scientists to effectively organise this increasing load. This data is high in volume, velocity, and veracity \cite{ali2}. Organisations have been overwhelmed with such a large amount of data, and are trying to tackle this by rapidly adopting Big Data Technology such as Hadoop, Spark and Flink as frameworks to efficiently store and utilise the data when needed.

\textbf{Mathematics}
Any good data scientist needs a strong foundation on both maths and statistics, as any data driven business such as Google or YouTube will need a data scientist to understand different approaches to constructing a statistical model \cite{ali4}.  Multivariate Calculus and Linear Algebra are key subjects used to create machine learning models, which has emerged as a key principle of data science. The fundamental thought process required for math also contributes to the logical thinking skills required for all aspects of computing roles.

\textbf{Data Manipulation/Visualisation}
Data visualisation is a key component of being a data scientist, as you need to be able to effectively communicate information for proposed solution.  Data visualisation sits in between technical analysis and visual storytelling. As big data becomes increasingly integral to business operations, visualisation has become a crucial tool in making sense of the vast volumes of data generated daily \cite{ali4}. The ability to break down complex data into smaller, digestible pieces by using a variety of visual aids like charts and graphs is key for data scientists to advance career wise.



\subsection{Skills: Roshni Vadassery : Software Development}

\textbf{Programming languages:} 
It is vital for software developers or engineers to be experienced users of one or more coding languages. By specialising in certain coding languages, it can positively impact the individuals future for any job opportunities in the industry. To be considered professional software developers by employers it is expected to know at least every coding language to a certain extent(basic level). Many industries can specialise in different programming languages, for example; some industries may use Java for mobile app development whereas for the design and creation of websites can be purely reliant on HTML.          
\textbf{Problem-solving skills:} 
A fundamental component of software development in the industry for individuals is to have excellent problem-solving skills. In the scope of software development, problem-solving is the logical process for identifying and solving any problems through software. This process involves the use of troubleshooting, correcting any bugs and finally resolving any arising issues or problems that occur throughout the process. In the software development working environment, it can be hectic when certain situations occur where clients or teams are struggling with their needs. Software developers are trained to adapt to these timed conditions of proposing the best solution for any problems.  

\textbf{Teamwork skills:} 
Teamwork in the software development industry is considered to be extremely efficient for future success. As a software developer, the job involves working collaboratively as a team to complete any given projects. The main goal is to collectively produce and cater to their company's desires and wants by creating the best final product. The benefits of teamwork include; increased creativity, reduced production time leading to better efficiency and the act of communication.     

\textbf{Data Structure and algorithms:} 
The notion of data structure and algorithms is significantly crucial for the hiring process in the industry. Software developers with a more advanced knowledge of data structures allows the individual to complete or perform any tasks with ease regarding calculations and data processing. It allows the software developer to create combinations by keeping the code clean and optimising the stored information for the eventual product. Some examples of common data structures include; arrays, stacks, linked lists, trees and queues. By knowing data structures and algorithms it allows for increased efficiency on performing operations on the storing and organisation of data. 

\textbf{Operating Systems:} 
All software developers should practically know the basics surrounding the operating systems. Operating systems is the software that configures a device's hardware which allows the applications to operate and run. The various types of the operating systems for computers are; ,macOS, Linux and Microsoft Windows. Whereas the operating systems for phones include; IOS and Android. It is advised in the industry to have a solid knowledge for all of these operating systems as it can improve testing procedures and the development of program codes. 

\textbf{Testing Procedures:} 
Testing procedures are generally important across the software development field as any software programs requires the use it before handing over the program to the company. It is considered as to be one of the final steps in the whole process as it cannot be released to consumers without any proper testing or judgement. The process involves the software developer to subject the code to discover any bugs and correct any applications allowing the product to properly function for future clients and consumers. The testing procedure can be split into three components; unit testing, integration testing and system testing.  

\textbf{Attention to detail:} 
Software development or coding in particular requires great accuracy and attention to detail. When working the the realm of software development, the main goal is to always be precise on removing any errors or debugging before any serious situations can occur. Software developers have to be able to focus for extended periods of time as creating software programs can be a long process. Without any accuracy, issues can arise where the end product of the program can be filled with bugs. In these situations, it is advised for the individual to always pay attention and be detail-oriented.

\textbf{Multitasking:} 
As a software developer, it is very possible for the individual to work on multiple projects under timed conditions. This can put on a lot of pressure on the individual but from the beginning of the process, software developers are trained to work simultaneously on many different projects. To become a successful software developer, it is essential to know and get use to the regular deadlines of submitting and working on multiple projects. By multitasking the individual can gain time management skills and excellent efficiency.  

\subsection{Skills: Lily Weng : Cyber Security}

\textbf{Coding}
Coding refers to creating computer programming code that allows the computers to output what is desired by the user. This is a significant skill to have in cybersecurity as it provides an understanding of the structure and architecture of a system meaning it is easier to defend it. This skill allows for thorough software examination to discover vulnerabilities and malicious codes to assist in detecting cyber criminals. The efficiency of detecting vulnerabilities which is achieved through having the skill of coding makes that employee more favourable when working in cyber security.

\textbf{Cloud security}
“Cloud security is a set of policies, methods, and technologies that protects the infrastructure, data, and applications that are cloud-based, whether the cloud be private, public, or a hybrid” \cite{l1}. There is an increasing number of organisations who use cloud infrastructure to store and run applications which require protection of cloud based systems and devices. This is because cyber criminals are also advancing their skills in hacking, making organisations’ digital platforms more vulnerable. Therefore, having security skills applicable in cloud security will always be in high demand in cyber security. 

\textbf{Security across various platforms}
Network security is “a set of rules and configurations designed to protect the integrity, confidentiality and accessibility of computer networks and data using both software and hardware technologies” \cite{l2}. The knowledge and skill of network security is crucial in all industries as the skill is used to protect computer networks from cyber criminals hacking into confidential and important information. However digital attacks do not only occur on computers. Cyber criminals can hack into various platforms such as operating systems, computer systems, mobile devices, cloud networks, and wireless networks. Therefore it is very important for cyber security experts to have deep knowledge of network security across these various platforms so they can efficiently protect critical systems and sensitive information. 

\textbf{Problem solving skills}
Working in cyber security naturally involves constantly protecting various technologies such as computer systems against digital threats from cyber criminals. Working in this field requires problem solving skills to be able to efficiently and effectively defeat each attack. Having problem solving skills proactively and reactively can fix the problem in a good timely manner to minimize the loss in reputation, financial benefits, and personal privacy. Additionally, this skill allows experts to take on complex cyber threats in a creative way across numerous digital environments, ensuring high security for the various technologies that contain confidential and important information. 

\textbf{Attention to Detail}
This creates a broad and in depth protection of the digital environment as detail oriented people are more alert to sudden changes and possible attacks. This  makes it more effective to detect vulnerabilities and risks. Not only is the attack detected early, having an eye to detail allows them to monitor the systems for long periods of time without losing focus; a valuable skill that is required for all industries that heavily depend on technology. The result of this is that they will become a dependable employee who will constantly been in high demand in industries such as cyber security. 

\textbf{Communication}
Communication is a vital skill in cybersecurity, as working in this field involves constantly communicating with employees who are in the same department or other departments. This is because all cyber security employees should be fully engaged, informed, and aware of the company's current status on their digital platforms, making it easier for these workers to navigate and investigate any vulnerabilities caused by cyber criminals. This can be only achieved through clear communication habits by the employees throughout the company. It is also crucial for cyber security employees to communicate clearly and effectively on different strategies, techniques, and approaches so the other employees working in the same department are aware of their actions and instantly can provide their assistance. This generates an organised workspace, creating a powerful defense for the company against cyber criminals. 

\textbf{Logical reasoning}
Logical reasoning is significantly useful in cyber security for both coding/hacking and detecting cyber attacks. Logic highly compliments the skill of coding as logic creates an organised and structured code that meets the requirements of output. It is evident that having logic will make coding more understand, eg, if statements, and while and for loops in python. As coding is crucial in cybersecurity, so is logic reasoning. Logic reasoning is also useful when monitoring digital systems for potential vulnerabilities and cyber attacks. It can allow for predictions of the criminal’s thought process in hacking, making it easier for cyber security experts to defend their systems from being attacked. Therefore, logic reasoning is important in cyber security as it leads to more organised coding as well as being able to defend digital attacks in early stages due to accurate predictions.

\textbf{Research and learning new information}
Whether it is researching to gain further insight into a cyber criminal’s logical thinking when hacking or for retaining new information, both these research motives are needed in cybersecurity as it displays a sense of resourcefulness. An employee working in cybersecurity must possess the positive attitude of continual learning as cybersecurity is constantly evolving and new technologies are rapidly developing. All employees need to be aware and up to date with the current and future innovations to ensure they can continue to use software to defend digital systems against cybercriminals.


%=======================================================================================

\newpage
\section{Level 2: Basic Technology}

Level 2 focuses on initial evaluation of the tech stack that is used by a selected company. All companies make use of a range of technologies, and these technologies need to work together. A tech stack is basically just this collection of technologies that collectively enable a company's systems. As an example, one of the most common technology stacks for supporting web servers is LAMP: Linux as the underlying operating system; Apache as a web server; MySQL as the supporting database; and Perl (or more recently PHP or Python) as the programming language.

Each student should choose a different tech stack and explain the role of each of the different technologies in that stack. Note that prior to researching your proposed tech stack and spending time writing about it, it might be a good idea to check with your tutor as to whether your chosen stack is suitable. (Target = $\sim$200-400 words per student).

\subsection{LAMP: Lily Weng}

LAMP is the acronym for Linux, Apache, MySQL, and PHP.  These four components are all free and open source software that contribute to LAMP’s architecture. LAMP is the tech stack of software that delivers high-performance web applications.

\textbf{Linux}
Linux is the operating system in the tech stack that forms most of the internet. Its role is the backbone of the techstack as the other components run on top of it. PHP and MySQL work most efficiently with Linux compared to other operating systems due to Linux’s flexibility and easy configuration options. 

\textbf{Apache}
Apache is the web server and the most popular server in the world. Its role is to allow accessibility of the web application to all of the public through using an URL. It does this by using the HTTP to process requests to deliver the information over the internet. Apache runs more than half of the websites on the internet. 

\textbf{MySQL}
MySQL (Structured Query Language) is the database. Its role is to store application data. The database has tables and inside the tables contains data that is collected. It can take in inputs from the user and store them into the database, eg, when logging into an account on the internet, MySQL checks if the username and password is correct. 

\textbf{PHP}
PHP (Hypertext Preprocessor) is the programming language used for writing the web applications. PHP is a crucial element in LAMP as the programming language can retrieve data from MySQL for the HTML. Its role is combining all the elements of the LAMP tech stack. 

When a user enters an URL in a browser, this sends a request to the web server(Apache). PHP is used to code what to display in the HTML. The code can access the database (MySQL) that checks if there is any data in the database to send back. After the code has collected the data that is needed for the HTML the operating system (Linux) is aware that the user’s incoming request is prepared. Then Apache executes the code which displays the HTML to the user’s browser. 


\subsection{ASP.NET: Dong Yoon Shin (Patrick)}

  ASP.NET, Active Server Pages Network Enabled Technologies, is an open web framework for creating outstanding websites and online apps with HTML, CSS, and JavaScript\cite{techst}. Microsoft created it to help programmers create dynamic web pages, apps, and services.

\textbf{ASP.NET MVC}
  ASP.NET MVC is a standard model-view-controller framework. It is built on top of ASP.NET and makes use of ASP.NET's APIs \cite{techst}. Developers may use ASP .NET MVC to create a web application by combining three roles, which are Model, View, and Controller. The MVC paradigm divides web applications into three levels of logic. Model, for example, a business layer, and view a display layer. 

\textbf{IIS}
  Internet Information Server (IIS) is a popular Microsoft web server for hosting and providing Internet-based services to ASP.NET and ASP Web applications\cite{ctan}. It is in charge of responding to user inquiries about the data supplied there as HTTP requests is mapped to directories depending on a site, an application, and a virtual directory.

\textbf{Angular frontend framework with TypeScript}
  For Angular application development, TypeScript is the primary language. It's a superset of JavaScript with type safety and tooling support at design time. Angular allows developers to quickly start new projects, update current projects, and add new components to their codebase by just typing a few lines into their terminal. \cite{angular}
  
\textbf{SQL Server}
  SQL Server is Microsoft's enterprise database that is built on SQL which is a programming language used to manage databases and query data.  It is designed to store data using a table structure based on rows, which allows data and functions to be linked while guaranteeing data security and consistency of web.


\subsection{Tech Stack: Roshni Vadassery}

\textbf{Ruby on rails:} Ruby on Rails(RoR) is a software code and environment that is written in the Ruby programming language. It is a web development framework which provides a variety of default structures for online services, databases and web apps. Ruby on Rails is known to work well alongside HTML, CSS, JavaScript and other web technologies. This tech stack is used to deliver productivity, easier development and maintenance for the overall framework of the web apps. The main aim of RoR is to optimise the programming work through its design philosophy. It has many features which each provide different services, these include; convention over configuration, easy programming language, MVC Architecture and many more.  Ruby on Rails is very cost effective as it is an open source structure meaning that is a web work structure that doesn't not need any sort of payment. It is versatile in the sense that this particular framework includes many productive, flexible and versatile applications. Ruby on Rails is particularly known for its fast pace programming process allowing the structure to be presented in a simpler manner for quicker changes to occur in the application process. Ruby on Rails is used by large companies like; Netflix, Github, Shopify and Airbnb. RoR has many benefits as it is widely used for its structure and aimed to build web applications. Ruby on Rails is the extended version of the technology Ruby. The technology Ruby is a programming language like Java which is used to build desktop applications. Through the technology Ruby, the framework of Ruby on Rails was built by the programming language to create web based applications. Ruby on Rails is designed to encompass both the front end and back end(unlike other programming languages) which explores the key characteristics of creating and building web based applications. 

\subsection{MEAN: Li Sheng}

The MEAN (MongoDB, Express.js, Angular.js, Node.js) stack is a JavaScript based framework for developing web applications. MEAN applications can be used in a variety of ways with a cross platform write once approach. While JavaScript is a fantastic modern language, its designed to focus on the front-end, meaning back-end problems with concurrency and performance at scale could occur.\cite{ali5}

\textbf{Angular.js}
At the top of the stack is Angular.js, a JavaScript MVW (Model View and Whatever) Framework. Angular.js extends HTML tags with metadata to create more dynamic and engaging experiences in comparison to purely HTML and JavaScript. Its features include form validation, localisation, and communication to the back-end service.

\textbf{Express.js / Node.js}
The next level down is Express.js, running on a Node.js server, where Express.js is a fast, unopinionated, minimalist web framework for Node.js. Express.js possess efficient models for URL routing, and handling HTTP requests and responses. Meaning that HTTP requests are received via the Angular.js frontend, and then connected to the Express.js functions that power the application \cite{ali5}. Those functions in turn employs MongoDB’s Node.js drivers to access and update the MongoDB database.

\textbf{MongoDB}
Any application that requires data storage and requires working in tandem with Angular, Express, and Node could utilise the MongoDB database. JSON documents created in the Angular.js frontend can be sent to the Express.js server, where it is processed and stored in MongoDB. The stored data allows full database security and cross cloud scalability.

%=======================================================================================

\newpage
\section{Level 3: Advanced Skills}

Level 3 focuses on more advanced technical skills (\LaTeX\ and Git) and analysis of linkages and relationships between the items in the company tech stack.

The following is a list of advanced Git and \LaTeX\ skills/features. Each student should select one pair of items from each list and demonstrate actual use of each item (either through activity in Git, or through including items in this report). (Target = $\sim$100 words per student for each feature).
\begin{itemize}
    \item Git
    \begin{itemize}
        \item Rebasing and Ignoring files
        \item Forking and Special files
        \item Resetting and Tags
        \item Reverting and Automated merges
        \item Hooks and Tags
    \end{itemize}
    \item \LaTeX\ 
    \begin{itemize}
        \item Cross-referencing and Custom commands
        \item Footnotes/margin notes and creating new environments
        \item Floating figures and editing style sheets
        \item Graphics and advanced mathematical equations
        \item Macros and hyperlinks
    \end{itemize}
\end{itemize}

\subsection{Advanced features: Li Sheng}

\textbf{Git – Forking and Special Files}
The forking workflow is fundamentally different to the usual Git workflow of using a single server sided repository as the central database, and instead opts for giving every developer their own server-side repository. As only the project maintainer can push towards the official repository, meaning that the maintainer can accept commits from any developer without providing them write access to the official codebase. \cite{ali6}

Special files offer a variety of formats and name extensions, such as README.md which explains the repository and CHANGELOG.md which explains notable updates, such are employed to improve repository management and developer interactions. \cite{ali7}


\textbf{Latex – Footnotes/Margin Notes and Creating New Environments}
The ability to add footnotes provides further context and detail to a block of text, adding more value and dynamics to the document. The \verb|\footnote []| command creates an automatically generated footnote with the numbers automatically ordered, whereas the \verb|\footnote[21]| command would create a footnote with an assigned value (21) \cite{ali8}.

Similarly, margin notes formats blocks of texts, creating a new environment by adding notes on either the right or the left of text blocks with the \verb|\marginpar [left text] {right text}| command. Alongside the \verb|\raggedright| and \verb|\raggedleft| commands changes the typesetting of margin notes \cite{ali9}.


\subsection{Advanced features: Lily Weng}

\textbf{Git - Rebasing and Ignoring Files}
“The rebase command in git is one of the main ways to integrate changes from one branch to another branch. Rebasing is the process of moving or combining a sequence of commits to a new base commit” \cite{l9}. By using the rebase command, the changes that were committed in one branch can be copied into a different branch. Git does this by creating new commits into the new branch and and apply them to the specific base. The primary usage of rebasing is to create a “linear project history”. Eg, a person decides to work on a feature branch of the project instead of the main branch, to keep information neat and understandable but also have the benefit of making it look like they are working on the main branch. This organised method is created by using the rebase command, which updates the latest commits of the feature branch to the main branch, effectively integrating the information together into one main branch.

There is no command in git that allows users to ignore files, instead users create a gitignore file. The ignored files are tracked in this gitignore file that is checked in at the root of the user’s repository. The user must manually edit and commit into the gitignore file to specify to git which files are intentionally untracked, therefore these files are ignored. “Some common examples of ignored files include: dependency caches, compiled code, build output directories, files generated at runtime, hidden system files, and personal IDE config files”  \cite{l11}.

\textbf{Latex - Cross-referencing and Custom commands}
In latex, cross referencing is done by assigning each element, such as a chart or picture, to a number which can later be used when the user wants to reference different parts of the document. The command \verb|\ref| is used to execute cross referencing. Cross referencing includes two steps. The user must save the elements with the appropriate number for reference, and then it must replace the \verb|\ref| with the appropriate number. The user must recompile the document twice to see the correct numbering of the elements in order for accurate cross referencing.

Although latex has an abundance of in-built commands which makes coding easier for users, new custom commands are essential as it is useful in minimising repetitive or complex formatting. The \verb|\newcommand| function is executed to create the new command, and two arguments are required; the name of the new compound and the definition of the command. Additionally, different types of commands can be made. Simple commands which is a basic commands with the name and definition of command. Its format is: \verb|\newcommand{\name}{\defintion{name}}|. Commands involving parameters can also be created where the name and definition of the command is given, as well as the number of parameters the new command accepts (n). Its format is : \verb|\newcommand{\name}{n}{\defintion{name}}|. Other commands include Commands with optional parameters and Overwriting existing commands.

\subsection{Advanced features: Roshni Vadassery}

\textbf{Git - Reverting and Automated merges}
Reverting in Git is basically considered as the undoing type/style of command. According to the standard undo operation, in normal terms, the commit would be completely removed from the project history, but the git revert command will undo a commit to return to the repository's commit history. To further explain, instead of completely removing the commit, using the git revert command it will create a brand new commit that will undo any changes to a published commit. By using the git revert command, it prevents Git from losing any project history.

To put it simply, merging in Git allows to integrate any branched history into one single branch. Specifically, the automated merge can be used when all merging requirements are completed, pull requests can be enabled to merge automatically. Unless there are any modifications in both commit sequences that clash, Git can automatically merge commits. Through the preparation to merge, Git will produce a merge commit.  The git merge command will determine the merge algorithm automatically by auto merging any of the separate histories.

\textbf{Latex - Graphics and advanced mathematical equations}
Latex provides a picture environment in which you can design and draw graphics and pictures. Although xfig may generate code for the image environment, the resulting graphics have significant restrictions, including the ability to replicate only particular slopes and circles. Currently, the best option is to utilise xfig to create postscript files, which have no such restrictions but do require a postscript printer or comparable. You should utilise the figure environment for any graphics you want to add so that LATEX can handle circumstances when, for example, you try to insert a half-page high graphic near the bottom of a page. With your choices, the figure environment will float the graphic to the top or bottom of the page, or to the next page.

For advanced mathematical equations and expressions, LATEX has two writing modes: inline math and display math. Inline math mode is used to write formulas that are within a paragraph or statement. Whereas, display math mode is used to write expressions that aren't part of a paragraph and are thus placed on their own separate lines. To write more complex equations or expressions, where you can insert or add an inline formula by typing the equation inside the Latex box. But for advanced mathematical equations, the amsmath package is recommended as it includes very powerful, flexible and new commands compared to the basic Latex.


\subsection{Advanced features: Dong Yoon Shin (Patrick Shin)}
\textbf{Git - Hooks and Tags}
Git hooks are scripts/functions that execute automatically in a Git repository when particular events occur \cite{patty2}. This allows Git's internal behaviour to be customised, as well as the triggering of customisable actions at crucial moments in the development process. By typing “cd .git/hooks/” on terminal to see the list. By typing the format of (file)/.git/hooks/(list of hooks), the script for specific event can be set. Furthermore, more can be done with Git hooks: A lint check for code style or syntax issues before committing, executing the test code before pushing and abort the push if it fails, and keeping the commit log consistent, you may even construct a commit statement automatically.

Git tags are used to identify certain commits and capture specific points in history. Tags are read-only commits because they are immutable. Tags are typically used to release software versions \cite{patty1}. Tags are divided into two categories: lightweight tags and annotated tags. Lightweight tag is a tag that just contains the tag name, such as version. Annotated tags can be signed with the tag creator's name, email, tagging date, and even tag messages.
"git tag [name]" is for generating Lightweight Tags. For example, "git tag v1.0" to create the tag v1.0, and "git tag -l "v1.0" to check the tag. For Annotated tags, use command ‘git tag -a [name] -m [messages]. The editor is opened automatically if the -m option is not specified. 'git tag -a v1.1 -m "second tag 1.1"' is an example of producing Annoated tag.


\textbf{Latex – Floating figures and editing style sheets}
Floats are containers for items in a document that cannot be broken across many pages. The floats "table" and "figure" are recognised by default in LaTeX, but you can define your own. In the document's preface, add "\usepackage{float}". Then define a new float using the following syntax: "\newfloat{type}{placement}{ext}[outer counter]", where type is the new name of the float, placement is t, b, p, or h ais letters, and ext is the file name extension of an auxiliary file for list of figures. The presence of the parameter is shown by the outside counter. The kind is the default name for the caption that appears at the beginning. “\floatname{type}{floatname}” can be used to change this, and“\thispagestyle” is a command for changing the float style.

Upper and lower cotton columns are available from LATEX in three different styles. The page style refers to the face-to-face format. "\pagestyle{style}" is the command. Plain, headings, and empty are the parameters that can be used in the style. Plain is a page layout that displays the page number in the centre of the footer without a header. Without a footer, a heading is a header with the name of the chapter and section. The term Empty refers to a page with no header or footer. To change the document's current page. Use the "\thispage{style}" command. This allows you to alter the page's style \cite{patty}.




%=======================================================================================

\newpage
\section{Level 4: Advanced Knowledge}

Level 4 focuses on analysing your particular tech stack and considering alternatives. Each student should consider the tech stack they described for Level 2, and then discuss each of the following points:
\begin{itemize}
    \item What are the strengths and limitations of this stack? (Target = $\sim$200 words).
    \item What alternatives exist, and under what situations might these alternatives be a better choice? (Target = $\sim$200 words).
\end{itemize}

\subsection{Advanced Knowledge: Li Sheng}

\textbf{Strengths of MEAN Stack}
Benefits from being based around a single uniform language (JavaScript) meaning its completely free, as well as allowing developers to create simple open-source solutions, and easily execute projects such as calendars, expense tracking, new aggregation sites and mapping and location finding applications. \cite{ali10}

The combination itself is flexible, scalable, and extensible, making it the perfect candidate for cloud hosting. Including its own web server so it can be deployed easily and be scaled on demand in response to usage spikes. \cite{ali11}

There is also no requirement to reformat the JSON format as the MEAN stack components Angular.js and Node.js both use JSON. MongoDB employs the JSON format while storing data, meaning no reformatting data is needed, being pivotal in support of larger projects. \cite{ali12}

\textbf{Limitations of MEAN Stack}
The MEAN stack is not recommended for large-scale applications. During heavy load scenarios, there may be a potential loss of records written by MongoDB. Furthermore, there are no established JavaScript guidelines for coding. \cite{ali11}

While it is flexible in dealing with usage spikes and MEAN stack-based changes, it is difficult to adapt a site to another approach after it has been created with the MEAN stack. \cite{ali10}


\textbf{MERN Stack as an Alternative}
The MERN stack is a common alternative to the MEAN stack, replacing Angular.js with React.js, making the development process much smoother and easier as React.js can handle rapidly changing data quickly. \cite{ali13}

The MEAN stack is backed by Google while the MERN stack is backed by Facebook.

MERN executes code development quickly due to its collection of dynamic user interfaces readily available in the library. Being the best in controlling and updating large dynamic JSON data that can smoothly navigate the frontend and backend. \cite{ali14}

React.js sometimes is preferred by web developers due to having no template, as it does not require them to take time to learn and practice templating language to automate the task of creating repetitive HTML or DOM elements. \cite{ali15}

React.js is isomorphic, meaning the same code can run on both server and the browser allowing businesses to create pages on the server when required.

Overall, the MERN stack is easier to learn, develops at a quicker rate, performs better in maintaining UI rendering, suitable for managing larger projects, and is preferred for the quick development for smaller applications. \cite{ali13}


\subsection{Advanced Knowledge: add student 2 name here}

Your text goes here

\subsection{Advanced Knowledge: add student 3 name here}

Your text goes here

\subsection{Advanced Knowledge: add student 4 name here}

Your text goes here



%=======================================================================================

\newpage

\bibliographystyle{apacite}
\bibliography{project}


\end{document}
\end{report}
