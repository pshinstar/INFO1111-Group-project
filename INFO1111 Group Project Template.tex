\documentclass[a4paper, 11pt]{report}
\usepackage{blindtext}
\usepackage[T1]{fontenc}
\usepackage[utf8]{inputenc}
\usepackage{titlesec}
\usepackage{fancyhdr}
\usepackage{geometry}

\usepackage[english]{babel}
\usepackage{apacite}

\geometry{ margin=30mm }
\counterwithin{subsection}{section}
\renewcommand\thesection{\arabic{section}.}
\renewcommand\thesubsection{\thesection\arabic{subsection}.}
\usepackage{tocloft}
\renewcommand{\cftchapleader}{\cftdotfill{\cftdotsep}}
\renewcommand{\cftsecleader}{\cftdotfill{\cftdotsep}}
\setlength{\cftsecindent}{2.2em}
\setlength{\cftsubsecindent}{4.2em}
\setlength{\cftsecnumwidth}{2em}
\setlength{\cftsubsecnumwidth}{2.5em}


\begin{document}
\titleformat{\section}
{\normalfont\fontsize{15}{0}\bfseries}{\thesection}{1em}{}
\titlespacing{\section}{0cm}{0.5cm}{0.15cm}
\titleformat{\subsection}
{\normalfont\fontsize{13}{0}\bfseries}{\thesubsection}{0.5em}{}
\titlespacing{\section}{0cm}{0.5cm}{0.15cm}

%=======================================================================================

\begin{titlepage}
\center 
\textbf{\huge INFO1111: Computing 1A Professionalism}\\[0.75cm]
\textbf{\huge 2022 Semester 1}\\[2cm]
\textbf{\huge Practice: Team Project Report}\\[3cm]

\textbf{\huge Submission number: ??}\\[0.75cm]
\textbf{\huge Team Members:}\\[0.75cm]
\textbf{\large
    \begin{tabular}{|p{0.5\textwidth}|p{0.3\textwidth}|p{0.2\textwidth}|}
        \hline
        Name & Student ID & Levels being attempted in this submission\\
        \hline
        Ali Sheng & 520470671 & e.g. 1,2 \\
        Dong (Patrick) Yoon Shin & 520271045 & ?? \\
        Lily Weng & 520474864 & ?? \\
        ?? & ?? & ?? \\
        \hline
    \end{tabular}
}\\[0.75cm]
\end{titlepage}

%=======================================================================================

\tableofcontents

%=======================================================================================

\newpage
\section*{General Instructions}

You should use this \LaTeX\ template to generate your team project report. Keep in mind the following key points:
\begin{itemize}
    \item When we assess your report, you are not given a mark. Instead we will indicate (separately, for each team member) whether each level is ''achieved''.
    \item In order to pass the unit, you must achieve at least level 1. 
    \item In order to achieve level 2, you must first have achieved level 1, and so on for each level up to level 4. This means that we will not assess a higher level until a lower level has been achieved (though we will review one level higher and give you feedback to help you in refining your work).
    \item Some parts of the report are completed as a team and other parts require each student to complete a different section. This means that for each submission, some members of the team may have completed their work for a given section, but other members may not. It also is therefore possible that some members of the team may achieve a specified level and other members of the team may not yet have achieved that level.
    \item Even if some members are completing their material for a given level, and others are not, your team members will still need to work together to edit and compile the report.  The only exception to this is where a member of the team has already achieved the level they are targeting in a previous submission and has decided to not attempt higher levels, and so is not contributing any further (this should be obvious because no level is indicated for that student on the cover page).
    \item When completing each section you should remove the explanation text and replace it with your material.
\end{itemize}

For each submission you will add new details to this report, and/or update previous sections (where previous work was not good enough to have achieved the relevant level). In particular:

\begin{itemize}
    \item \textbf{General:} For each submission, each student can attempt up to 2 levels. You must also successfully achieve each lower level before you can be assessed at a higher level. For example, in the first submission you might attempt only level 1, but not be successful in achieving that level. You then reattempt level 1 and add in level 2 in the second submission and are successful in achieving level 1 but not level 2. For the third and final submission you could then attempt level 2, or levels 2 and 3 - or even just choose to not submit anything further and remain at level 1).
    \item \textbf{Submission 1:} You should complete at least the material for level 1 (since achieving level 1 is required to pass the unit). Each member of the team can also optionally choose to complete the material for level 2.\\
    \textit{Note 1: If you do not complete the level 2 information then you obviously cannot achieve level 2 at this stage. This does not stop you from attempting level 2 in Deliverable 2 or 3, but it will make it more difficult to achieve the higher levels later in the semester.}
    \textit{Note 2: To be able to achieve Level 1 in submission one your team has to achieve level 1 in the group component (Section 1.1) and you have to achieve Level 1 in the individual component (i.e. your assigned section 1.2, 1.3, 1.4 or 1.5)}
    \item \textbf{Submission 2:} Each member of your team will complete additional sections, but because you are submitting a single document, you need to work together to compile your results together and generate the final submission.\\
    If you did not achieve level 1 in your first submission, then you should revise the material for level 1 based on the feedback, and optionally you can also complete level 2.\\
    If you achieved level 1 in your first submission, then each team member can optionally complete the material for levels 2 and 3.
    \textit{Note: If you do not achieve level 1 with this submission then the highest level you will be able to achieve in the final submission will be level 2. If you achieve level 1, but not level 2, with this submission then the highest level you will be able to achieve with the final submission is level 3.}
    \item \textbf{Submission 3:} Again, you can correct sections where you did not achieve the specified level in the previous submission, and you complete additional sections.\\
    If you still have not achieved level 1, then you should revise the material for level 1 based on the feedback, and again optionally you can also complete level 2.\\
    For those at level 1, you can choose to complete the material for levels 2 and 3.\\
    For those at level 2, you can choose to complete the material for levels 3 and 4.\\
    For those at level 3, you can choose to complete the material for level 4.
\end{itemize}

Whilst the team project is just that -- a team project -- it has been designed to also allow different members of the team to achieve different outcomes. We do expect you to work together as a team. If you do come across problems working together then the first step should be to discuss this with your tutor. Note: If you are having problems you should approach your tutor as soon as you can to make them aware of the difficulties you are having with your team.

Finally, you should also ensure that any resources you use are suitably referenced, and references are included into the reference list at the end of this document. You should use APA 6th reference style \cite{apa6}.

%=======================================================================================

\newpage
\section{Level 1: Basic Skills}

Level 1 focuses on basic technical skills (related to \LaTeX\ and Git) and the types of skills used in different computing jobs.

\subsection{Developing industry skills}

This section is completed as a team.\\
Throughout your Computing degree we will help you learn a range of new skills. Once you graduate however you will need to continue to learn new languages, new tools, new applications, etc. For this section you need to identify 5 approaches you can take to this continual learning. You should then put these in order from most effective to least effective, and then explain the circumstances in which each approach might be appropriate. (Target = $\sim$100 words per skill = $\sim$500 words total).

\textbf{Efficient Workspace}

With the field of computing perpetually accelerating with new technology, a high level of self-discipline and autonomy is required to maintain a solid level of professionalism and to keep learnt skills in practice. As working at home has become a common phenomenon after 2020s, constructing an efficient workspace is key to improving productivity. Usually, keeping distractions out of reach and building a strong and focused mindset is the classic go-to, but that requires much self-control and would only come with time. Some other options would be going to cafes, libraries and even shared workspaces, as being in an environment where others are working encourages concentration.

\textbf{Personal Projects}

Besides being a library of codes displaying the skillset of a computer scientist, personal projects assist in the important task of keeping the myriad of languages used in practice. While everyone has their own opinion on the best language e.g. Python, C/C++, JavaScript for whatever task at hand, their variety of contradicting syntax requires practice to prevent re-learning of the language, wasting time and stalling advancement. Furthermore, the best way to show yourself and potential employers the level of your abilities in a particular coding language and particular field is to create a personal project.

\textbf{Developing Goal}

\textbf{Self learning}
Working in computer science requires the attitude of continual learning to be aware and up to date with the rapidly evolving innovations relating to technology. This allows employees to continue maintaining their valuable status for the company and prevent skills from becoming outdated. It is common for all workers in computer science to execute self learning to some level as it shows the employer that the candidate has full capability to perform personal projects. It also shows the employer the candidate has significant desire to learn, creating an impressive impression when choosing successful candidates. Therefore the positive action of continual self learning in the computer science industry is seen as an essential and required skill to have, as it maintains their skills that are applicable with the latest technologies and also generates a positive impression amongst employers


\subsection{Skills: ?? : ??}

This section is completed individually. Each member of the team should independently complete a separate copy of this section.\\
You should begin by allocating to each team member a different major to focus on (i.e. one of: Computer Science; Data Science; Software Development; Cyber Security). \textit{If you have a fifth member, then your tutor will suggest a fifth topic to cover}. You should then undertake research into the typical practical skills that you believe would be most important to someone who graduates with this major and is then working in industry. You should list the 8 skills that you believe are most important and for each one give a short explanation as to why you feel it is important. (Target = $\sim$100 words per skill $\sim$800 words total per student).

\subsection{Skills: Dong Yoon Shin : Computer Science}

\textbf{Understanding Human Error}
While human process machine learning, programming, and creating algorithms, resulting in severe system dangers. Human error is a significant factor to consider in complicated safety-critical systems since it contributes the most to overall system risk. Mistakes can never be eliminated since, in comparison to computers, the human aspect makes forecasting, avoiding, and unpredictability more difficult. For example, collected data, coding. Understanding how human error occurs helps to the prevention of large system risks and the reduction of human-induced errors, such as those made by co-workers in industry. As a result, more qualified outcomes will be obtained with fewer significant problems.

\textbf{Creativity}
Professional computer scientists demand creativity to break free from old patterns or habits of thought and open up new possibilities. Unlike mathematicians, computer scientists must be able to benefit from an infinite number of possibilities and solution paths, which necessitates inventiveness. Most modern companies require someone who can think of new ways to expand existing solutions, such as algorithms. Innovation leads to the development of all computer systems in an industry as a result of creative ideas. Creativity that can be paired with technology that can handle them freely is necessary in the reconfiguration and exploitation of current computer systems., especially to construct and deal with new systems, software, requires logical thinking and inventiveness.

\textbf{Mathematics Skills}
Computer scientists must have a good understanding of mathematical theory, including calculus, statistics, linear algebra, some physics concepts, and any other advanced mathematical methods, as mathematics is a general necessity for any computer factor. This is because, while mathemtics is an important intellectual tool in programming, computing is becoming increasingly important in solving mathematical issues. Furthermore, Mathematical Concepts are Required in Many Computer Science Disciplines, such as the use of algorithms and the provision of Analytical Skills. As a result, mathemtics abilities are required to comprehend computer systems used in the modern IT business, such as those used by Google and Microsoft.

\textbf{Programming Skills}
Python, R, Java, Javascirpt, C, and more programming languages are available to computer scientists. Computer scientists research the theory of computation and the design of software systems, hence they are in charge of investigating these processes from a computer science standpoint. They must, however, be able to solve problems with computers and apply solutions while coding. The primary function of a computer scientist is design, not construction. To build an algorithm, computer scientist need to be familiar with a variety of programming languages. In addition, by learning computer languages, you will be able to assist teams with their responsibilities.

\textbf{Analytical Skills}
To think critically, examine data, make difficult judgments, and solve complicated issues, computer scientists need great analytical skills. A computer scientist's main task is to identify a problem and design a solution to address it. This necessitates strong analytical abilities, which will allow computer sicentists to grasp the situation at hand and evaluate numerous options before selecting the one that best fit the needs of all stakeholders. Computer scientists utilize two analytical skills during the task data analysis and experimenting with technical applications. These abilities result in accurate data and outcomes, as well as the greatest results for clients and businesses.

\textbf{Problem Solving Skills}
One of the most important skills for computer scientists is problem solving. Majors in computer science must be able to tackle complicated issues in a systematic and logical manner, which necessitates the use of mathematics, creativity, and other skills. The reason for this is that a computer scientist must select which project he or she is working on and how to take a concept and turn it into a reality. For this, computer scientists must be able to comprehend problems that can be solved by humans and computers, as well as write specialised syntax to solve programmes. This can assist the team in determining what they need to do, as well as what tools and resources they will require. Overall, the team's efficiency has improved.




\subsection{Skills: Li Sheng : Data Science}


\textbf{Communication}
While strong communication skills are crucial to essentially all lines of work, I feel it is especially important to data scientists. The role requires communicating information to team members/partners that may not have a data/computing background, as well as needing to understand the ‘problem’ that needs to be solved. As data scientists would usually work in teams, good communication would allow them to mesh into various team dynamics easily. Furthermore, as pure intuition would only go so far, good interpersonal skills would help probe stakeholders for more data, information, and key insights.

\textbf{Business Acumen}
Adding onto communication skills, business knowledge adds to the effectiveness of communication, making the process of problem solving much more efficient. As data scientists usually work to solve issues for businesses, business insight assists in understanding the key objectives and goals of the business, and thus the ability to create cost effective, easy to implement solutions which would ensure broad adoption. Furthermore, it is helpful in promoting business growth and exploring new business opportunities from a more data focused angle.

\textbf{Critical/Structured Thinking}
The core of data science requires deep curiosity and critical thinking, to solve problems and to find solutions, as data on its own does not mean much without processing and interpretation from data scientists. Alongside critical thinking, storytelling skills are also important, for example, the score of a live basketball game, on one side, it is displayed in the form of a table, the other side, as a bar chart, whereas the bar chart obviously being more intuitive, as it is human nature to prefer a more visual medium e.g. infographics, and it being the job of the data scientist display intricate data in a medium digestible to the average person.

\textbf{Programming}
Being a data scientist means communicating with machines, with programming languages such as Python, R and C/C++ as their tools. While they do not need to become the best at programming like software engineers do, they will need to be comfortable with it. The main role of programming languages for data scientists is to organise, analyse and visualise data, so having a library of different languages helps tackle this task from multiple angles. This also assists in filling basic team roles e.g. programming, frontend/backend engineering, which all computing roles would entail.

\textbf{Machine Learning/Deep Learning}
For data scientists working with mountains of organised and unorganised data, machines/deep learning builds predictive models and algorithms which effectively streamlines the organisation and prediction of data. This relatively emerging school of technology is not necessary for a data scientist but is a skill which would improve employability by automating the entire process by cleaning data and removing redundancies. Machine learning is also extremely malleable, being used in Fraud/Risk Detection Management, Healthcare, Airline Route Planning, Spam Filtering, Facial/Voice Recognition System, Interactive Voice Response, Comprehensive Language and Document Recognition and Translation.

\textbf{Big Data}
The growing ubiquity of the internet, social media and IoT has led to a sudden boom in the rate of data being generated, being up to 2.5 quintillion per day, new and innovative methods are required for data scientists to effectively organise this increasing load. This data is high in volume, velocity, and veracity. Organisations have been overwhelmed with such a large amount of data, and are trying to tackle this by rapidly adopting Big Data Technology such as Hadoop, Spark and Flink as frameworks to efficiently store and utilise the data when needed 

\textbf{Mathematics}
Any good data scientist needs a strong foundation on both maths and statistics, as any data driven business such as Google or YouTube will need a data scientist to understand different approaches to constructing a statistical model.  Multivariate Calculus and Linear Algebra are key subjects used to create machine learning models, which has emerged as a key principle of data science. The fundamental thought process required for math also contributes to the logical thinking skills required for all aspects of computing roles.

\textbf{Data Manipulation/Visualisation}
Data visualisation is a key component of being a data scientist, as you need to be able to effectively communicate information for proposed solution.  Data visualisation sits in between technical analysis and visual storytelling. As big data becomes increasingly integral to business operations, visualisation has become a crucial tool in making sense of the vast volumes of data generated daily. The ability to break down complex data into smaller, digestible pieces by using a variety of visual aids like charts and graphs is key for data scientists to advance career wise.



\subsection{Skills: add student 3 name here : Software Development}
fs
Your text goes here

\subsection{Skills: Lily Weng : Cyber Security}

\textbf{Coding}
Coding refers to creating computer programming code that allows the computers to output what is desired by the user. This is a significant skill to have in cybersecurity as it provides an understanding of the structure and architecture of a system meaning it is easier to defend it. This skill allows for thorough software examination to discover vulnerabilities and malicious codes to assist in detecting cyber criminals. The efficiency of detecting vulnerabilities which is achieved through having the skill of coding makes that employee more favourable when working in cyber security.

\textbf{Cloud security}
“Cloud security is a set of policies, methods, and technologies that protects the infrastructure, data, and applications that are cloud-based, whether the cloud be private, public, or a hybrid” (Acronis, n.d.). There is an increasing number of organisations who use cloud infrastructure to store and run applications which require protection of cloud based systems and devices. This is because cyber criminals are also advancing their skills in hacking, making organisations’ digital platforms more vulnerable. Therefore, having security skills applicable in cloud security will always be in high demand in cyber security. 

\textbf{Security across various platforms}
Network security is “a set of rules and configurations designed to protect the integrity, confidentiality and accessibility of computer networks and data using both software and hardware technologies” (Forcepoint, n.d.). The knowledge and skill of network security is crucial in all industries as the skill is used to protect computer networks from cyber criminals hacking into confidential and important information. However digital attacks do not only occur on computers. Cyber criminals can hack into various platforms such as operating systems, computer systems, mobile devices, cloud networks, and wireless networks. Therefore it is very important for cyber security experts to have deep knowledge of network security across these various platforms so they can efficiently protect critical systems and sensitive information. 

\textbf{Problem solving skills}
Working in cyber security naturally involves constantly protecting various technologies such as computer systems against digital threats from cyber criminals. Working in this field requires problem solving skills to be able to efficiently and effectively defeat each attack. Having problem solving skills proactively and reactively can fix the problem in a good timely manner to minimize the loss in reputation, financial benefits, and personal privacy. Additionally, this skill allows experts to take on complex cyber threats in a creative way across numerous digital environments, ensuring high security for the various technologies that contain confidential and important information. 

\textbf{Attention to Detail}
This creates a broad and in depth protection of the digital environment as detail oriented people are more alert to sudden changes and possible attacks. This  makes it more effective to detect vulnerabilities and risks. Not only is the attack detected early, having an eye to detail allows them to monitor the systems for long periods of time without losing focus; a valuable skill that is required for all industries that heavily depend on technology. The result of this is that they will become a dependable employee who will constantly been in high demand in industries such as cyber security. 

\textbf{Communication}
Communication is a vital skill in cybersecurity, as working in this field involves constantly communicating with employees who are in the same department or other departments. This is because all cyber security employees should be fully engaged, informed, and aware of the company's current status on their digital platforms, making it easier for these workers to navigate and investigate any vulnerabilities caused by cyber criminals. This can be only achieved through clear communication habits by the employees throughout the company. It is also crucial for cyber security employees to communicate clearly and effectively on different strategies, techniques, and approaches so the other employees working in the same department are aware of their actions and instantly can provide their assistance. This generates an organised workspace, creating a powerful defense for the company against cyber criminals. 

\textbf{Logical reasoning}
Logical reasoning is significantly useful in cyber security for both coding/hacking and detecting cyber attacks. Logic highly compliments the skill of coding as logic creates an organised and structured code that meets the requirements of output. It is evident that having logic will make coding more understand, eg, if statements, and while and for loops in python. As coding is crucial in cybersecurity, so is logic reasoning. Logic reasoning is also useful when monitoring digital systems for potential vulnerabilities and cyber attacks. It can allow for predictions of the criminal’s thought process in hacking, making it easier for cyber security experts to defend their systems from being attacked. Therefore, logic reasoning is important in cyber security as it leads to more organised coding as well as being able to defend digital attacks in early stages due to accurate predictions.

\textbf{Research and learn new information}
Whether it is researching to gain further insight into a cyber criminal’s logical thinking when hacking or for retaining new information, both these research motives are needed in cybersecurity as it displays a sense of resourcefulness. An employee working in cybersecurity must possess the positive attitude of continual learning as cybersecurity is constantly evolving and new technologies are rapidly developing. All employees need to be aware and up to date with the current and future innovations to ensure they can continue to use software to defend digital systems against cybercriminals.


%=======================================================================================

\newpage
\section{Level 2: Basic Technology}

Level 2 focuses on initial evaluation of the tech stack that is used by a selected company. All companies make use of a range of technologies, and these technologies need to work together. A tech stack is basically just this collection of technologies that collectively enable a company's systems. As an example, one of the most common technology stacks for supporting web servers is LAMP: Linux as the underlying operating system; Apache as a web server; MySQL as the supporting database; and Perl (or more recently PHP or Python) as the programming language.

Each student should choose a different tech stack and explain the role of each of the different technologies in that stack. Note that prior to researching your proposed tech stack and spending time writing about it, it might be a good idea to check with your tutor as to whether your chosen stack is suitable. (Target = $\sim$200-400 words per student).

\subsection{Tech Stack: add student 1 name here}

Your text goes here

\subsection{Tech Stack: add student 2 name here}

Your text goes here

\subsection{Tech Stack: add student 3 name here}

Your text goes here

\subsection{Tech Stack: add student 4 name here}

Your text goes here


%=======================================================================================

\newpage
\section{Level 3: Advanced Skills}

Level 3 focuses on more advanced technical skills (\LaTeX\ and Git) and analysis of linkages and relationships between the items in the company tech stack.

The following is a list of advanced Git and \LaTeX\ skills/features. Each student should select one pair of items from each list and demonstrate actual use of each item (either through activity in Git, or through including items in this report). (Target = $\sim$100 words per student for each feature).
\begin{itemize}
    \item Git
    \begin{itemize}
        \item Rebasing and Ignoring files
        \item Forking and Special files
        \item Resetting and Tags
        \item Reverting and Automated merges
        \item Hooks and Tags
    \end{itemize}
    \item \LaTeX\ 
    \begin{itemize}
        \item Cross-referencing and Custom commands
        \item Footnotes/margin notes and creating new environments
        \item Floating figures and editing style sheets
        \item Graphics and advanced mathematical equations
        \item Macros and hyperlinks
    \end{itemize}
\end{itemize}

\subsection{Advanced features: add student 1 name here}

Explain your use of the advanced Git and \LaTeX\ features. 

\subsection{Advanced features: add student 2 name here}

Explain your use of the advanced Git and \LaTeX\ features. 

\subsection{Advanced features: add student 3 name here}

Explain your use of the advanced Git and \LaTeX\ features. 

\subsection{Advanced features: add student 4 name here}

Explain your use of the advanced Git and \LaTeX\ features. 



%=======================================================================================

\newpage
\section{Level 4: Advanced Knowledge}

Level 4 focuses on analysing your particular tech stack and considering alternatives. Each student should consider the tech stack they described for Level 2, and then discuss each of the following points:
\begin{itemize}
    \item What are the strengths and limitations of this stack? (Target = $\sim$200 words).
    \item What alternatives exist, and under what situations might these alternatives be a better choice? (Target = $\sim$200 words).
\end{itemize}

\subsection{Advanced Knowledge: add student 1 name here}

Your text goes here

\subsection{Advanced Knowledge: add student 2 name here}

Your text goes here

\subsection{Advanced Knowledge: add student 3 name here}

Your text goes here

\subsection{Advanced Knowledge: add student 4 name here}

Your text goes here



%=======================================================================================

\newpage

\bibliographystyle{apacite}
\bibliography{main}

Dewani, R 2020, 14 Skills Required To Become A Data Scientist in 2020, Analytics Vidhya.

Udacity Team 2020, Top 8 Skills You Need to be a Data Scientist, Udacity.

Van Loon, R 2017, Top 6 Skills You Need to Master to Become a Data Scientist, Simplilearn.com.

What Skills Do You Need to Become a Data Scientist? 2020, Springboard Blog, viewed 24 March 2022, <https://www.springboard.com/blog/data-science/data-science-skills/>.

Udacity Team 2020, Top 8 Skills You Need to be a Data Scientist, Udacity.

Acronis. (n.d.). What is cloud-based security and how does it work?. Retrieved from https://www.acronis.com/en-sg/articles/cloud-based-security/?gclid=Cj0KCQjw_4-SBhCgARIsAAlegrU3PdUpW3sKINn6dAPOEJZqpc_gNXoqrS-SbcbSq71owwEzX_r41KcaAsuBEALw_wcB

Forcepoint. (n.d.). What is Network Security?. Retrieved from https://www.forcepoint.com/cyber-edu/network-security

Dat, M. (n.d.). 7 Must-Have Skills for Cybersecurity Success. Retrieved from https://startacybercareer.com/six-skills-needed-for-success-in-cyber-security/

Krakoff, S. (n.d.) Top Cybersecurity Skills in High Demand. Retrieved from https://online.champlain.edu/blog/top-cybersecurity-skills-in-high-demand


\end{document}
\end{report}
